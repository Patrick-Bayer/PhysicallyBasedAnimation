\documentclass[
	11pt, 
	DIV10,
	a4paper, 
	oneside, 
	headings=normal, 
	captions=tableheading,
	final, 
	numbers=noenddot
]{scrartcl}


\usepackage{lipsum}
\usepackage{graphicx}


\title{Simulating Visual Geometry}
\subtitle{\vspace{0.5cm}Seminar: Current Topics in Physically-Based Animation}
\author{Patrick Bayer}


\begin{document}
\maketitle
\tableofcontents


\section{Introduction}
	\begin{itemize}
		\item Ultimate goal in virtual worlds: What you see is what you simulate
		\item Unfortunately not so easy to fulfill because of performance restrictions in real time environments $\rightarrow$ two or three different representations for simulated objects (visual mesh, simulation mesh, collection of convex shapes for collision handling)
		\item Possible mismatch of collision behaviour and visual appearance
		\item New method with a single representation for both, simulation and collision handling, and an almost identical visualization mesh
		\item Objects created as triangel or quad meshes that are not prepared for physically-based animation $\rightarrow$ embed the visual mesh in a simulation mesh or attach it to a more general structure
		\item Use of a simulation mesh close to the visual mesh/directly derived from the visual mesh
		\item Convex shapes as primitives
		\item General graph to connect the primitives
		\item Primitives are treated as oriented particles and represented with convex polyhedra
		\item Oriented particle method for simulation
	\end{itemize}
\section{Related Work}
	\begin{itemize}
		\item Combining the rigid body formulation with a deformable model
		\item Fracture and tearing algorithms
	\end{itemize}
\subsection{Simulation Models}
	\begin{itemize}
		\item General, non-conforming unstructured mesh
		\item Unified solver based on position based dynamics
		\item Oriented particle method
		\item Shape matching
	\end{itemize}
\subsection{Embedding of a Visual Mesh}
	\begin{itemize}
		\item Embed visual mesh in a tetrahedal mesh and a regular grid
		\item Blend the transformations defined in nearby particles by their position and orientation
	\end{itemize}
\subsection{Fracturing and Tearing}
	\begin{itemize}
		\item Use of pre-fractured models
		\item Pre-defined fracture patterns at the impact location
		\item Representing objects and fracture patterns with convex decompositions
	\end{itemize}
\section{The Method}

\subsection{Physical Mesh Creation}
	\begin{itemize}
		\item Input: Visual mesh with triangle and quad faces potentially intersecting each other
		\item Physical properties that cannot be derived from the geometry are defined by the user
		\item Create volumetric convex primitives if needed
		\item Average positions of the vertices are used for simulation
	\end{itemize}
\subsection{The Simulation Method}
	\begin{itemize}
		\item Oriented particle approach
		\item Shape matching
		\item Standard collision detection
	\end{itemize}
\subsubsection{Oriented Particle Method}
	\begin{itemize}
		\item Oriented particles store rotation, spin and the usual linear attributes (postion, velocity, ...)
		\item overall look and feel of objects $>$ accurate reproduction of their small scale behaviour 
		\item easy tuning of physical attributes
		\item full control of object behaviour
		\item based on generalization of PBD and shape matching
	\end{itemize}
\subsubsection{Shape Matching}

\subsection{Deforming Primitives}
	\begin{itemize}
		\item Oriented particle method only modifies the location and orientation of primitves $\rightarrow$ potential gaps and collision artifacts
		\item Primitives have to stay convex in order to perform collision handling and fracturing
		\item Deform primitives in using the local affine deformation matrices	
	\end{itemize}

\subsection{The Visualization Mesh}
	\begin{itemize}
		\item Vertices of different primitives are joined for visualization using global IDs
		\item Compute the resulting visual positions with help of the average positions of joined vertices
	\end{itemize}	

\subsection{Plastic Deformation}
	\begin{itemize}
		\item Treat the entire object as a rigid body if there are rigid parts
		\item Rearrange the primitives only when they are plastically deformed
		\item Deform objects if the relative normal velocity is above a threshold defined in the material
		\item Deformation offset = relative normal velocity * time step
	\end{itemize}

\subsection{Subdivision, Fracturing and Tearing}
	\begin{itemize}
		\item[] \textbf{Fracture}
		\item Connected set of convex polyhedra as fracture pattern
		\item Align fracture pattern with the impact location
		\item Compute the cuts of all the primitves of the objects against all fracture cells $\rightarrow$ new independent objects
		\item Modification: Keep all the resulting pieces in the same object
		\item Always apply fracture patterns in the undeformed configuration
		\item[] \textbf{Tearing}
		\item Use a threshold for the distance between connected primitives in order to check whether the connection is broken
		\item Use only a subset of the links including the links introduced by applying a fracture pattern
	\end{itemize}

\section{Results}

\section{Conclusion and Future Work}


\bibliographystyle{alpha}
\bibliography{references}

\end{document}          
